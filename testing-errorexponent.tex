\documentclass[10pt]{article}
\def\withcolors{1}
\def\withnotes{1}
\def\withindex{0}
\usepackage[T1]{fontenc}
\usepackage[utf8]{inputenc}

%% Eye-candy
\usepackage{lmodern}
\usepackage{xspace}                                     % Smart spacing with \xspace
\usepackage[protrusion=true,expansion=true]{microtype}  % Improve font rendering

% Striking out text
\usepackage[normalem]{ulem}

%% Math
\usepackage{amsfonts,amsmath,amssymb, amsthm, mathtools}
\usepackage{thm-restate}
\usepackage{dsfont} % For the indicator symbol

% Algorithm environment
\usepackage{algorithmicx,algpseudocode,algorithm}

% Colors (with names)
\usepackage[usenames,dvipsnames,table]{xcolor}

% Quotes: \blockquote command
\usepackage{csquotes}

% Relative sizes for text
\usepackage{relsize}

% Bibliography
%\usepackage[numbers]{natbib}

% Required for the table of results
\usepackage{multirow}
\usepackage{chngpage} % allows for temporary adjustment of side margins

% For the commands such as \capitalisewords
\usepackage{mfirstuc}

% Graphics
\usepackage{tikz}
\usetikzlibrary{arrows}
\usetikzlibrary{calc,decorations.pathmorphing,patterns}

% For indexing
\ifnum\withindex=1
  \usepackage{makeidx}
  \usepackage{ifthen}
  \newcommand\indexed[2][]{\ifthenelse{\equal{#1}{}}{#2\index{#2}}{#2\index{#1}}}
  \makeindex %%%% Enable indexing
\fi
%%\usepackage{showidx} % To debug; does not play well with hyperref

% References and links
\usepackage[colorlinks,citecolor=blue,bookmarks=true,linktocpage]{hyperref}
\usepackage{aliascnt}
\usepackage[numbered]{bookmark}

% Full pages
\usepackage{fullpage}

% Titling
\usepackage{titling}

% Compressed lists
\usepackage[shortlabels]{enumitem}
  \setitemize{noitemsep,topsep=3pt,parsep=2pt,partopsep=2pt} % Uncomment for compact item lists
  \setenumerate{itemsep=1pt,topsep=2pt,parsep=2pt,partopsep=2pt}
  \setdescription{itemsep=1pt}
  
% Package for todo notes.
\ifnum\withnotes=1
  \usepackage[colorinlistoftodos,textsize=scriptsize]{todonotes}
\fi

% Verbatim inputs and code
\usepackage{verbatim}

% Resizable parentheses that work (without the space between \left(#1\right)
\usepackage{mleftright} % \mleft( #1 \mright)

\makeatletter
\@ifundefined{theorem}{%
  % Theorems (each with its own style, all same counter). Cf. http://ftp.math.purdue.edu/mirrors/ctan.org/macros/latex/contrib/hyperref/doc/manual.pdf, p.17
  \theoremstyle{plain} %% Style
  	\newtheorem{theorem}{Theorem}%[section]
  	\newaliascnt{coro}{theorem}
  	  \newtheorem{corollary}[coro]{Corollary}
  	\aliascntresetthe{coro}
  	\newaliascnt{lem}{theorem}
  		\newtheorem{lemma}[lem]{Lemma}
  	\aliascntresetthe{lem}
  	\newaliascnt{clm}{theorem}
  		\newtheorem{claim}[clm]{Claim}
	\aliascntresetthe{clm}
	\newaliascnt{fact}{theorem}
 	 	\newtheorem{fact}[theorem]{Fact}
	\aliascntresetthe{fact}
  	\newtheorem*{unnumberedfact}{Fact}
  \newaliascnt{prop}{theorem}
  		\newtheorem{proposition}[prop]{Proposition}
	\aliascntresetthe{prop}
	\newaliascnt{conj}{theorem}
  		\newtheorem{conjecture}[conj]{Conjecture}
	\aliascntresetthe{conj}
  \theoremstyle{remark} %% Style
  	\newtheorem{remark}[theorem]{Remark}
  	\newtheorem{question}[theorem]{Question}
  	\newtheorem*{notation}{Notation}
 	 \newtheorem{example}[theorem]{Example}
  \theoremstyle{definition} %% Style
  	\newaliascnt{defn}{theorem}
 		 \newtheorem{definition}[defn]{Definition}
 	 \aliascntresetthe{defn}
}{}
\makeatother
\providecommand*{\lemautorefname}{Lemma} % For \autoref{} to know the name of lemmas
\providecommand*{\clmautorefname}{Claim}
\providecommand*{\propautorefname}{Proposition}
\providecommand*{\coroautorefname}{Corollary}
\providecommand*{\defnautorefname}{Definition}
\newenvironment{proofof}[1]{\begin{proof}[Proof of {#1}]}{\end{proof}}

%% \email{} command
\providecommand{\email}[1]{\href{mailto:#1}{\nolinkurl{#1}\xspace}}

%% Remarks and notes
\ifnum\withcolors=1
  \newcommand{\new}[1]{{\color{red} {#1}}} % new
  \newcommand{\newer}[1]{{\color{blue} {#1}}} % even newer
  \newcommand{\newest}[1]{{\color{orange} {#1}}} % even even newer
  \newcommand{\newerest}[1]{{\color{blue!10!black!40!green} {#1}}} % you get the idea.
  \newcommand{\ccolor}[1]{{\color{RubineRed}#1}} % Clement
\else
  \newcommand{\new}[1]{{{#1}}}
  \newcommand{\newer}[1]{{{#1}}}
  \newcommand{\newest}[1]{{{#1}}}
  \newcommand{\newerest}[1]{{{#1}}}
  \newcommand{\ccolor}[1]{{#1}}
\fi

\ifnum\withnotes=1
  \newcommand{\cnote}[1]{\par\ccolor{\textbf{C: }\sf #1}} % Clement
  \newcommand{\todonote}[2][]{\todo[size=\scriptsize,color=red!40,#1]{#2}}  
	\newcommand{\questionnote}[2][]{\todo[size=\scriptsize,color=blue!30]{#2}}
	\newcommand{\todonotedone}[2][]{\todo[size=\scriptsize,color=green!40]{$\checkmark$ #2}}
	\newcommand{\todonoteinline}[2][]{\todo[inline,size=\scriptsize,color=orange!40,#1]{#2}}  
  \newcommand{\marginnote}[1]{\todo[color=white,linecolor=black]{{#1}}}
\else
  \newcommand{\cnote}[1]{}
  \newcommand{\todonote}[2][]{\ignore{#2}}
	\newcommand{\questionnote}[2][]{\ignore{#2}}
	\newcommand{\todonotedone}[2][]{\ignore{#2}}
	\newcommand{\todonoteinline}[2][]{\ignore{#2}}
  \newcommand{\marginnote}[1]{\ignore{#1}}
\fi
\newcommand{\ignore}[1]{\leavevmode\unskip} % eat unnecessary spaces before
\newcommand{\cmargin}[1]{\questionnote{\ccolor{#1}}} % Clement

% Shortcuts
\newcommand{\eps}{\ensuremath{\varepsilon}\xspace}
\newcommand{\Algo}{\ensuremath{\mathcal{A}}\xspace} % Algorithm A
\newcommand{\Tester}{\ensuremath{\mathcal{T}}\xspace} % Testing algorithm T
\newcommand{\Learner}{\ensuremath{\mathcal{L}}\xspace} % Learning algorithm L
\newcommand{\property}{\ensuremath{\mathcal{P}}\xspace} % Property P
\newcommand{\class}{\ensuremath{\mathcal{C}}\xspace} % Class C
\newcommand{\eqdef}{\stackrel{\rm def}{=}}
\newcommand{\eqlaw}{\stackrel{\mathcal{L}}{=}}
\newcommand{\accept}{\textsf{ACCEPT}\xspace}
\newcommand{\fail}{\textsf{FAIL}\xspace}
\newcommand{\reject}{\textsf{REJECT}\xspace}
\newcommand{\opt}{{\textsc{opt}}\xspace}
\newcommand{\half}{\frac{1}{2}}
\newcommand{\domain}{\ensuremath{\Omega}\xspace} % Domain of a distribution (default notation)
\newcommand{\distribs}[1]{\Delta\!\left(#1\right)} % Domain of a distribution (default notation)
\newcommand{\yes}{{\sf{}yes}\xspace}
\newcommand{\no}{{\sf{}no}\xspace}
\newcommand{\dyes}{{\cal Y}}
\newcommand{\dno}{{\cal N}}

% Complexity
\newcommand{\littleO}[1]{{o\mleft( #1 \mright)}}
\newcommand{\bigO}[1]{{O\mleft( #1 \mright)}}
\newcommand{\bigOSmall}[1]{{O\big( #1 \big)}}
\newcommand{\bigTheta}[1]{{\Theta\mleft( #1 \mright)}}
\newcommand{\bigOmega}[1]{{\Omega\mleft( #1 \mright)}}
\newcommand{\bigOmegaSmall}[1]{{\Omega\big( #1 \big)}}
\newcommand{\tildeO}[1]{\tilde{O}\mleft( #1 \mright)}
\newcommand{\tildeTheta}[1]{\operatorname{\tilde{\Theta}}\mleft( #1 \mright)}
\newcommand{\tildeOmega}[1]{\operatorname{\tilde{\Omega}}\mleft( #1 \mright)}
\providecommand{\poly}{\operatorname*{poly}}

% Influence
\newcommand{\totinf}[1][f]{{\mathbf{Inf}[#1]}}
\newcommand{\infl}[2][f]{{\mathbf{Inf}_{#1}(#2)}}
\newcommand{\infldeg}[3][f]{{\mathbf{Inf}_{#1}^{#2}(#3)}}

% Sets and indicators
\newcommand{\setOfSuchThat}[2]{ \left\{\; #1 \;\colon\; #2\; \right\} } 			% sets such as "{ elems | condition }"
\newcommand{\indicSet}[1]{\mathds{1}_{#1}}                                              % indicator function
\newcommand{\indic}[1]{\indicSet{\left\{#1\right\}}}                                             % indicator function
\newcommand{\disjunion}{\amalg}%\coprod, \dotcup...

% Distance
\newcommand{\dtv}{\operatorname{d}_{\rm TV}}
\newcommand{\kl}{\operatorname{KL}}
\newcommand{\dhell}{\operatorname{d_{\rm{}H}}}
\newcommand{\hellinger}[2]{{\dhell\mleft({#1, #2}\mright)}}
\newcommand{\kldiv}[2]{{\kl\mleft({#1 \,\|\, #2}\mright)}}
\newcommand{\kolmogorov}[2]{{\operatorname{d_{\rm{}K}}\mleft({#1, #2}\mright)}}
\newcommand{\totalvardistrestr}[3][]{{\dtv^{#1}\mleft({#2, #3}\mright)}}
\newcommand{\totalvardist}[2]{\totalvardistrestr[]{#1}{#2}}
%\newcommand{\chisquarerestr}[3][]{{\operatorname{d}^{#1}_{\chi^2}\mleft({#2 \mid\mid #3}\mright)}}
\newcommand{\chisquare}[2]{{\chi^2\mleft({#1 \mid\mid #2}\mright)}}
\newcommand{\dist}[2]{\operatorname{dist}\mleft({#1, #2}\mright)}

% Restriction (functions, sequences, etc.)
\newcommand\restr[2]{{% we make the whole thing an ordinary symbol
  \left.\kern-\nulldelimiterspace % automatically resize the bar with \right
  #1 % the function
  \vphantom{\big|} % pretend it's a little taller at normal size
  \right|_{#2} % this is the delimiter
  }}

% Probability
\newcommand{\proba}{\Pr}
\newcommand{\probaOf}[1]{\proba\!\left[\, #1\, \right]}
\newcommand{\probaCond}[2]{\proba\!\left[\, #1 \;\middle\vert\; #2\, \right]}
\newcommand{\probaDistrOf}[2]{\proba_{#1}\left[\, #2\, \right]}

% Support of a distribution/function
\newcommand{\supp}[1]{\operatorname{supp}\!\left(#1\right)}

% Expectation & variance
\newcommand{\expect}[1]{\mathbb{E}\!\left[#1\right]}
\newcommand{\expectCond}[2]{\mathbb{E}\!\left[\, #1 \;\middle\vert\; #2\, \right]}
\newcommand{\shortexpect}{\mathbb{E}}
\newcommand{\var}{\operatorname{Var}}

% Distributions
\newcommand{\uniform}{\ensuremath{\mathcal{U}}}
\newcommand{\uniformOn}[1]{\ensuremath{\uniform\!\left( #1 \right) }}
\newcommand{\geom}[1]{\ensuremath{\operatorname{Geom}\!\left( #1 \right)}}
\newcommand{\bernoulli}[1]{\ensuremath{\operatorname{Bern}\!\left( #1 \right)}}
\newcommand{\bern}[2]{\ensuremath{\operatorname{Bern}^{#1}\!\left( #2 \right)}}
\newcommand{\binomial}[2]{\ensuremath{\operatorname{Bin}\!\left( #1, #2 \right)}}
\newcommand{\poisson}[1]{\ensuremath{\operatorname{Poisson}\!\left( #1 \right) }}
\newcommand{\gaussian}[2]{\ensuremath{ \mathcal{N}\!\left(#1,#2\right) }}
\newcommand{\gaussianpdf}[2]{\ensuremath{ g_{#1,#2}}}
\newcommand{\betadistr}[2]{\ensuremath{ \operatorname{Beta}\!\left( #1, #2 \right) }}

% Norms
\newcommand{\norm}[1]{\lVert#1{\rVert}}
\newcommand{\normone}[1]{{\norm{#1}}_1}
\newcommand{\normtwo}[1]{{\norm{#1}}_2}
\newcommand{\norminf}[1]{{\norm{#1}}_\infty}
\newcommand{\abs}[1]{\left\lvert #1 \right\rvert}
\newcommand{\dabs}[1]{\lvert #1 \rvert}
\newcommand{\dotprod}[2]{ \left\langle #1,\xspace #2 \right\rangle } 			% <a,b>
\newcommand{\ip}[2]{\dotprod{#1}{#2}} 			% shortcut

\newcommand{\vect}[1]{\mathbf{#1}} 			% shortcut

% Ceiling and floor
\newcommand{\clg}[1]{\left\lceil #1 \right\rceil}
\newcommand{\flr}[1]{\left\lfloor #1 \right\rfloor}

% Common sets
\newcommand{\R}{\ensuremath{\mathbb{R}}\xspace}
\newcommand{\C}{\ensuremath{\mathbb{C}}\xspace}
\newcommand{\Q}{\ensuremath{\mathbb{Q}}\xspace}
\newcommand{\Z}{\ensuremath{\mathbb{Z}}\xspace}
\newcommand{\N}{\ensuremath{\mathbb{N}}\xspace}
\newcommand{\cont}[1]{\ensuremath{\mathcal{C}^{#1}}}

% Oracles and variants
\newcommand{\ICOND}{{\sf INTCOND}\xspace}
\newcommand{\EVAL}{{\sf EVAL}\xspace}
\newcommand{\CDFEVAL}{{\sf CEVAL}\xspace}
\newcommand{\STAT}{{\sf STAT}\xspace}
\newcommand{\SAMP}{{\sf SAMP}\xspace}
\newcommand{\COND}{{\sf COND}\xspace}
\newcommand{\PCOND}{{\sf PAIRCOND}\xspace}
\newcommand{\ORACLE}{{\sf ORACLE}\xspace}

%% Terminology
\newcommand{\pdfsamp}{dual\xspace}
\newcommand{\cdfsamp}{cumulative dual\xspace}
\newcommand{\Pdfsamp}{\expandafter\capitalisewords\expandafter{\pdfsamp}}
\newcommand{\Cdfsamp}{\expandafter\capitalisewords\expandafter{\cdfsamp}}

% L_p norms
\newcommand{\lp}[1][1]{\ell_{#1}}

% Convolution
\DeclareMathOperator{\convolution}{\ast}

%% Terminology
\newcommand{\D}{\ensuremath{D}}
\newcommand{\distrD}{\ensuremath{\mathcal{D}}}
\newcommand{\birge}[2][\D]{\Phi_{#2}(#1)}
\newcommand{\iid}{i.i.d.\xspace}

% Sign
\DeclareMathOperator{\sign}{sgn}

%% Roman numerals
\makeatletter
\newcommand{\rom}[1]{\romannumeral #1}
\newcommand{\Rom}[1]{\expandafter\@slowromancap\romannumeral #1@}
\newcommand{\century}[2][th]{\Rom{#2}\textsuperscript{#1}}
\makeatother

% Hyperref and \autoref{} -- names
\renewcommand{\sectionautorefname}{Section} % To have "Section 5" instead of "section 5" with \autoref{}
\renewcommand{\chapterautorefname}{Chapter} % To have "Chapter 5" instead of "chapter 5" with \autoref{}
\renewcommand{\subsectionautorefname}{Section} % To have "Section 5" instead of "subsection 5" with \autoref{}
\renewcommand{\subsubsectionautorefname}{Section} % To have "Section 5" instead of "subsubsection 5" with \autoref{}
\def\algorithmautorefname{Algorithm}


%%%%%%%%%%%%%%%%%%%%%%%%%%%%%%%%%%%%%%%%%%%%%%%%%%%%%%%%%%%%%%%%%
% Add author and title info to PDF (and handles multiple authors)
%%%%%%%%%%%%%%%%%%%%%%%%%%%%%%%%%%%%%%%%%%%%%%%%%%%%%%%%%%%%%%%%%
\makeatletter
  \AtBeginDocument{
  \begingroup
  \toks@={}%
  \toksdef\toks@@=2 %
  \toks@@={}%
  \long\def\@ReturnFiFi#1#2\fi\fi{\fi\fi#1}%
  \def\scan@author#1#2 \and#3\@nil{%
  \ifx\\#3\\%
    \ifcase#1 %
      \toks@={#2}%
    \else
      \ifnum#1>1 %
        \toks@=\expandafter{%
          \the\expandafter\toks@\expandafter,\expandafter\space
          \the\toks@@
        }%
      \fi
      \toks@=\expandafter{\the\toks@\space and #2}%
    \fi
    \else
      \ifcase#1 %
        \toks@={#2}%
        \@ReturnFiFi{%
          \scan@author1#3\@nil
        }%
      \else
        \ifnum#1>1 %
          \toks@=\expandafter{%
            \the\expandafter\toks@\expandafter,\expandafter\space
            \the\toks@@
          }%
      \fi
      \toks@@={#2}%
      \@ReturnFiFi{%
        \scan@author2#3\@nil
      }%
    \fi
  \fi
  }%
  \expandafter\expandafter\expandafter\scan@author
  \expandafter\expandafter\expandafter0%
  \expandafter\@author\space\and\@nil
  \edef\x{\endgroup
  \noexpand\hypersetup{pdfauthor={\the\toks@}}%
  }%
  \x
  }
\makeatother


\usepackage{filecontents}

\renewcommand{\eqdef}{\coloneqq}
\usepackage{utopia}
\newcommand{\p}{\mathbf{p}}
\newcommand{\q}{\mathbf{q}}
\newcommand{\errproba}{\delta}
\newcommand{\ns}{n}
\newcommand{\errexp}{\mathcal{E}}


\title{A short note on simple hypothesis testing, Hellinger distance, and Chernoff information}
\date{August, 2023}

\begin{document}
\begin{flushleft}\sf\footnotesize
\makeatletter
\@date~- \today \hfill \@title
\makeatother
\end{flushleft}
\vspace{5mm}

The goal of this short note is to explain the relation between two ``folklore'' results on simple hypothesis testing, and, quite crucially, how they square with each other. Thanks to \href{https://www.ee.ntu.edu.tw/bio1.php?id=1080917}{Hao-Chung Cheng} for illuminating discussions.\bigskip

For two \emph{fixed} probability distributions $\p,\q\in\distribs{\domain}$ over a known arbitrary domain $\Omega$, we write $\Psi(\p,\q,\errproba)$ for the sample complexity of deciding, with probability of error at most $\errproba$, which of these two distributions a sequence of i.i.d. samples from an (unknown) probability distribution $\q\in\{\p_0,\p_1\}$ originates from: specifically, given a uniform prior\footnote{One can generalize this to a non-uniform prior; the characterization of the error exponent as Chernoff information will remain the same, as the prior ``disappears'' asymptotically.} on $(\p_0,\p_1)$, the error of a test $T\colon \Omega^\ns \to \{0,1\}$ taking $\ns$ samples is
\begin{equation}
	\label{eq:err:proba:total}
	\errproba \eqdef \frac{1}{2}\probaDistrOf{\p_0}{T(X_1,\dots, X_\ns) = 1} + \frac{1}{2}\probaDistrOf{\p_1}{T(X_1,\dots, X_\ns) = 0}
\end{equation}
It is well-known (and described in one of these ``short notes'') that $\Psi(\p,\q,\errproba)$ is characterized by the \emph{squared Hellinger distance} between $\p$ and $\q$:
\begin{fact}[Sample complexity of simple hypothesis testing]
	\label{fact:sample:complexity:sht}
	For any $\p_0,\p_1$ and $\errproba\in(0,1]$, 
	\[
		\Psi(\p_0,\p_1,\errproba) = \bigTheta{\frac{\log(1/\errproba)}{\hellinger{\p_0}{\p_1}^2}}
	\]
	where $\hellinger{\p_0}{\p_1} = \frac{1}{\sqrt{2}} \normtwo{\sqrt{\p_0}-\sqrt{\p_1}}$ is the Hellinger distance.
\end{fact} 
Flipping things around, one could ask, given $\ns$ samples, what the best achievable probability of error $\errproba^\ast$ (as in~\eqref{eq:err:proba:total})) is as a function of $\ns,\p_0,\p_1$. Let's write $\errexp_\ns(\p_0,\p_1) \eqdef \frac{1}{\ns}\ln\frac{1}{\errproba^\ast(\ns,\p_0,\p_1)}$ for this ``finite-sample'' \emph{error exponent}, so that 
\begin{equation}
\errproba^\ast = e^{-\ns \errexp_\ns(\p_0,\p_1)}
\end{equation}
 Then,~\autoref{fact:sample:complexity:sht} appears to state that
\begin{equation}
	\label{eq:error:exp:hell}
\errexp_\ns(\p_0,\p_1) = \bigTheta{\hellinger{\p_0}{\p_1}^2}\,.
\end{equation}
This is, however, quite annoying, as a classsical result in information theory and statistics, the Chernoff bound,\footnote{No, not \emph{that} Chernoff bound.} states that $\lim_{\ns\to\infty} \errexp_\ns(\p_0,\p_1) = C(\p_0,\p_1)$, where 
\begin{equation}
	C(\p_0,\p_1) \eqdef - \min_{\lambda\in[0,1]}\ln \sum_{x\in\domain} \p_0(x)^\lambda \p_1(x)^{1-\lambda} 
\end{equation}
is the \emph{Chernoff information} between $\p_0$ and $\p_1$ (with the straightforward generalization if $\domain$ is not a discrete domain). \emph{Annoying}, because $C(\p_0,\p_1)$ and $\hellinger{\p_0}{\p_1}^2$ are clearly not the same thing, and having two different (and seemingly \emph{wildly} different) things characterize the same quantity is very confusing at best. So, erm, \textbf{how come?}

\section{Hellinger distance:~\eqref{eq:error:exp:hell} is not wrong\dots}
We first provide a self-contained proof of~\eqref{eq:error:exp:hell}; actually, of a stronger version of it, with explicit constants. This is adapting and combining the contents from another of these short notes, ``A short note on distinguishing discrete distributions'' (2017), and~\cite[Theorem 4.7]{BarYossef:02}.
\begin{lemma}
	\label{lemma:errexponent:hell}
For any $\p_0,\p_1$, and $\ns \geq 1$, we have
$
	e^{2\ns\ln(1-\hellinger{\p_0}{\p_1}^2)} \leq 2\errproba^\ast(\ns,\p_0,\p_1) \leq e^{\ns\ln(1-\hellinger{\p_0}{\p_1}^2)}
$, 
\emph{i.e.,}
\begin{equation}
			-\ln(1-\hellinger{\p_0}{\p_1}^2) + \frac{\ln 2}{\ns} \leq \errexp_\ns(\p_0,\p_1)  \leq -2\ln(1-\hellinger{\p_0}{\p_1}^2) + \frac{\ln 2}{\ns}\,,
\end{equation}
which implies~\eqref{eq:error:exp:hell}.
\end{lemma}
\begin{proof}
By the standard interpretation of total variation distance as characterization of the minimal sum of Type I and Type II errors, we have that 
 \begin{equation}
  	\label{eq:tv:typeI:II}
	1-2\errproba^\ast(\ns,\p_0,\p_1) = \totalvardist{\p_0^{\otimes n}}{\p_1^{\otimes \ns}}^2
\end{equation}
since $\errproba^\ast(\ns,\p_0,\p_1)$ was defined in~\eqref{eq:err:proba:total} as the optimal average error probability when distinguishing $\p_0$ and $\p_1$ from $\ns$ i.i.d.\, samples. So our task boils down to establishing good enough upper and lower bounds on $\totalvardist{\p_0^{\otimes n}}{\p_1^{\otimes \ns}}^2$.

 
To do so, we will rely on the following two relatively straightforward facts about Hellinger distance, with respect to total variation:
  \begin{equation}
  	\label{eq:tv:hellinger}
    1-\sqrt{1-\totalvardist{\p_0}{\p_1}^2} \leq \hellinger{\p_0}{\p_1}^2 \leq \totalvardist{\p_0}{\p_1}
  \end{equation}
 and products (tensoring):
\begin{equation}
	\label{eq:hellinger:product}
\hellinger{\p_0^{\otimes \ns}}{\p_1^{\otimes \ns}}^2 = 1 - \mleft(1-\hellinger{\p_0}{\p_1}^2\mright)^{\ns}\,.
\end{equation}
By~\eqref{eq:hellinger:product}, this implies 
$
    \hellinger{\p_0^{\otimes n}}{\p_1^{\otimes \ns}}^2 = 1 - \mleft(1-\hellinger{\p_0}{\p_1}^2\mright)^\ns = 1-e^{\ns\ln(1-\hellinger{\p_0}{\p_1}^2)} 
$, 
and therefore , by~\eqref{eq:tv:hellinger},
\begin{equation}
\totalvardist{\p_0^{\otimes n}}{\p_1^{\otimes \ns}}^2 \geq 1-e^{\ns\ln(1-\hellinger{\p_0}{\p_1}^2)}
\end{equation}
\noindent Conversely, from the lower bound from~\eqref{eq:tv:hellinger} and using~\eqref{eq:hellinger:product}, we get
\begin{align}
	\totalvardist{\p_0^{\otimes n}}{\p_1^{\otimes \ns}}^2 
	\leq 1- \mleft( 1-\hellinger{\p_0^{\otimes \ns}}{\p_1^{\otimes \ns}}^2 \mright)^2
	= 1 - \mleft(1-\hellinger{\p_0}{\p_1}^2\mright)^{2\ns}
	= 1 - e^{2\ns\ln(1-\hellinger{\p_0}{\p_1}^2)}
\end{align}
and so, combining the two and recalling~\eqref{eq:tv:typeI:II}, we finally get
\begin{equation}
e^{2\ns\ln(1-\hellinger{\p_0}{\p_1}^2)} \leq 2\errproba^\ast(\ns,\p_0,\p_1) \leq e^{\ns\ln(1-\hellinger{\p_0}{\p_1}^2)}
\end{equation}
as claimed.
 \end{proof}

\section{\dots{} yet Chernoff is correct.}
To square~\autoref{lemma:errexponent:hell} with the Chernoff bound, which states that
\begin{equation}
	\lim_{\ns\to \infty}\errexp_\ns(\p_0,\p_1) = C(\p_0,\p_1)
\end{equation}
we need to argue that, while maybe not \emph{equal}, $-\ln(1-\hellinger{\p_0}{\p_1}^2)$ and $C(\p_0,\p_1)$ are always within a factor 2 of each other. Basically, that constant factors \emph{do}, after all, matter.

The first observation is to rewrite $1-\hellinger{\p_0}{\p_1}^2$ in an equivalent (and standard-ish) form involving the \emph{Bhattacharyya coefficient},
\begin{equation}
	\operatorname{BC}(\p_0,\p_1) \eqdef \sum_{x\in\domain} \sqrt{\p_0(x) \p_1(x)}\,.
\end{equation}
Namely, one can check that 
$
1-\hellinger{\p_0}{\p_1}^2 = 1 - \operatorname{BC}(\p_0,\p_1)$. This is very convenient, as now we want to compare
\[
	-\ln(1-\hellinger{\p_0}{\p_1}^2)
	= -\ln \operatorname{BC}(\p_0,\p_1)
\]
to
\[
	C(\p_0,\p_1) 
	= -\min_{\lambda\in[0,1]} \ln \sum_{x\in\domain} \p_0(x)^\lambda\p_1(x)^{1-\lambda} 
	= - \ln  \min_{\lambda\in[0,1]} \sum_{x\in\domain} \p_0(x)^\lambda\p_1(x)^{1-\lambda}\,. 
\]
Getting rid of the logarithms, it would be enough to show that 
$\min_{\lambda\in[0,1]} \sum_{x\in\domain} \p_0(x)^\lambda\p_1(x)^{1-\lambda}$
and 
$
\sum_{x\in\domain} \sqrt{\p_0(x) \p_1(x)}
$
are within a quadratic factor of each other. Big if true! And, fortunately, true. 
\begin{lemma}[Skewed Bhattacharyya coefficients are quadratically related]
For any $\p_0,\p_1$ and $\lambda\in[0,1]$, we have
\begin{equation}
		\mleft(\sum_{x\in\domain} \sqrt{\p_0(x) \p_1(x)}\mright)^2 \leq \sum_{x\in\domain} \p_0(x)^\lambda\p_1(x)^{1-\lambda} \leq 1\,.
\end{equation}
In particular, we have
\begin{equation}
\operatorname{BC}(\p_0,\p_1)^2 \leq \min_{\lambda\in[0,1]} \operatorname{BC}_\lambda(\p_0,\p_1) \leq \operatorname{BC}(\p_0,\p_1)
\end{equation}
where $\operatorname{BC}_\lambda(\p_0,\p_1) = \sum_{x\in\domain} \p_0(x)^\lambda\p_1(x)^{1-\lambda}$ denotes the \emph{$\lambda$-skewed Bhattacharyya coefficient}.
\end{lemma}
\begin{proof}
Fix any $\lambda \in(0,1)$ (the cases $\lambda\in\{0,1\}$ being immediate). First, we have
\[
\sum_{x\in\domain} \p_0(x)^\lambda\p_1(x)^{1-\lambda}
 = \sum_{x\in\domain} \p_1(x)\cdot \mleft(\frac{\p_0(x)}{\p_1(x)}\mright)^\lambda
 \leq \mleft(\sum_{x\in\domain} \p_1(x)\cdot \frac{\p_0(x)}{\p_1(x)}\mright)^\lambda
 = \mleft(\sum_{x\in\domain} \p_0(x)\mright)^\lambda = 1
\] 
using Jensen's inequality for the concave function $x\mapsto x^\lambda$. 

Second, let's use H\"older's (the generalized version, with 3 vectors) with exponents $(2/(1-\lambda), 2, 2/\lambda, 2)$, which satisfy $\frac{1-\lambda}{2}+\frac{1}{2}+\frac{\lambda}{2}=1$. We have
\begin{align*}
\sum_{x\in\domain} \p_0(x)^{1/2}\p_1(x)^{1/2}
 &= \sum_{x\in\domain} \p_0(x)^{\frac{1-\lambda}{2}}\cdot \p_0(x)^{\frac{\lambda}{2}}\p_1(x)^{\frac{1-\lambda}{2}}\cdot \p_1(x)^{\frac{\lambda}{2}}\\
 &\leq \mleft(\sum_{x\in\domain} \p_0(x)\mright)^{\frac{1-\lambda}{2}}\mleft(\sum_{x\in\domain} \p_0(x)^{\lambda}\p_1(x)^{1-\lambda} \mright)^{\frac{1}{2}}\mleft(\sum_{x\in\domain} \p_1(x)\mright)^{\frac{\lambda}{2}} \tag{H\"older}\\
 &= \mleft(\sum_{x\in\domain} \p_0(x)^{\lambda}\p_1(x)^{1-\lambda} \mright)^{\frac{1}{2}}\,,
\end{align*}
concluding the proof.
\end{proof}
This readily implies that, for every $\p_0,\p_1$,
\begin{equation}
-\ln(1-\hellinger{\p_0}{\p_1}^2) \leq C(\p_0,\p_1) \leq -2\ln(1-\hellinger{\p_0}{\p_1}^2)
\end{equation}
and sanity is restored.


%%%%%%%%%% Bibliography
\begin{filecontents}{references1.bib}
@phdthesis{BarYossef:02,
  author = {Bar-Yossef, Ziv},
  title = {The Complexity of Massive Data Set Computations},
  school = {UC Berkeley},
  year = {2002},
  note = {Adviser: Christos Papadimitriou. Available at~\url{http://webee.technion.ac.il/people/zivby/index_files/Page1489.html}.}
}
\end{filecontents}
%%%%%%%%%%%%%%%%%%%%%%%%%%%%%%%%%%%%%%%%%%%%%%%%%%%%%%%%%%%%%%
\bibliographystyle{alpha}
\bibliography{references1}
\end{document}
