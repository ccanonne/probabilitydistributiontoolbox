\documentclass[10pt]{article}
\def\withcolors{1}
\def\withnotes{1}
\def\withindex{0}
\usepackage[T1]{fontenc}
\usepackage[utf8]{inputenc}

%% Eye-candy
\usepackage{lmodern}
\usepackage{xspace}                                     % Smart spacing with \xspace
\usepackage[protrusion=true,expansion=true]{microtype}  % Improve font rendering

% Striking out text
\usepackage[normalem]{ulem}

%% Math
\usepackage{amsfonts,amsmath,amssymb, amsthm, mathtools}
\usepackage{thm-restate}
\usepackage{dsfont} % For the indicator symbol

% Algorithm environment
\usepackage{algorithmicx,algpseudocode,algorithm}

% Colors (with names)
\usepackage[usenames,dvipsnames,table]{xcolor}

% Quotes: \blockquote command
\usepackage{csquotes}

% Relative sizes for text
\usepackage{relsize}

% Bibliography
%\usepackage[numbers]{natbib}

% Required for the table of results
\usepackage{multirow}
\usepackage{chngpage} % allows for temporary adjustment of side margins

% For the commands such as \capitalisewords
\usepackage{mfirstuc}

% Graphics
\usepackage{tikz}
\usetikzlibrary{arrows}
\usetikzlibrary{calc,decorations.pathmorphing,patterns}

% For indexing
\ifnum\withindex=1
  \usepackage{makeidx}
  \usepackage{ifthen}
  \newcommand\indexed[2][]{\ifthenelse{\equal{#1}{}}{#2\index{#2}}{#2\index{#1}}}
  \makeindex %%%% Enable indexing
\fi
%%\usepackage{showidx} % To debug; does not play well with hyperref

% References and links
\usepackage[colorlinks,citecolor=blue,bookmarks=true,linktocpage]{hyperref}
\usepackage{aliascnt}
\usepackage[numbered]{bookmark}

% Full pages
\usepackage{fullpage}

% Titling
\usepackage{titling}

% Compressed lists
\usepackage[shortlabels]{enumitem}
  \setitemize{noitemsep,topsep=3pt,parsep=2pt,partopsep=2pt} % Uncomment for compact item lists
  \setenumerate{itemsep=1pt,topsep=2pt,parsep=2pt,partopsep=2pt}
  \setdescription{itemsep=1pt}
  
% Package for todo notes.
\ifnum\withnotes=1
  \usepackage[colorinlistoftodos,textsize=scriptsize]{todonotes}
\fi

% Verbatim inputs and code
\usepackage{verbatim}

% Resizable parentheses that work (without the space between \left(#1\right)
\usepackage{mleftright} % \mleft( #1 \mright)

\makeatletter
\@ifundefined{theorem}{%
  % Theorems (each with its own style, all same counter). Cf. http://ftp.math.purdue.edu/mirrors/ctan.org/macros/latex/contrib/hyperref/doc/manual.pdf, p.17
  \theoremstyle{plain} %% Style
  	\newtheorem{theorem}{Theorem}%[section]
  	\newaliascnt{coro}{theorem}
  	  \newtheorem{corollary}[coro]{Corollary}
  	\aliascntresetthe{coro}
  	\newaliascnt{lem}{theorem}
  		\newtheorem{lemma}[lem]{Lemma}
  	\aliascntresetthe{lem}
  	\newaliascnt{clm}{theorem}
  		\newtheorem{claim}[clm]{Claim}
	\aliascntresetthe{clm}
	\newaliascnt{fact}{theorem}
 	 	\newtheorem{fact}[theorem]{Fact}
	\aliascntresetthe{fact}
  	\newtheorem*{unnumberedfact}{Fact}
  \newaliascnt{prop}{theorem}
  		\newtheorem{proposition}[prop]{Proposition}
	\aliascntresetthe{prop}
	\newaliascnt{conj}{theorem}
  		\newtheorem{conjecture}[conj]{Conjecture}
	\aliascntresetthe{conj}
  \theoremstyle{remark} %% Style
  	\newtheorem{remark}[theorem]{Remark}
  	\newtheorem{question}[theorem]{Question}
  	\newtheorem*{notation}{Notation}
 	 \newtheorem{example}[theorem]{Example}
  \theoremstyle{definition} %% Style
  	\newaliascnt{defn}{theorem}
 		 \newtheorem{definition}[defn]{Definition}
 	 \aliascntresetthe{defn}
}{}
\makeatother
\providecommand*{\lemautorefname}{Lemma} % For \autoref{} to know the name of lemmas
\providecommand*{\clmautorefname}{Claim}
\providecommand*{\propautorefname}{Proposition}
\providecommand*{\coroautorefname}{Corollary}
\providecommand*{\defnautorefname}{Definition}
\newenvironment{proofof}[1]{\begin{proof}[Proof of {#1}]}{\end{proof}}

%% \email{} command
\providecommand{\email}[1]{\href{mailto:#1}{\nolinkurl{#1}\xspace}}

%% Remarks and notes
\ifnum\withcolors=1
  \newcommand{\new}[1]{{\color{red} {#1}}} % new
  \newcommand{\newer}[1]{{\color{blue} {#1}}} % even newer
  \newcommand{\newest}[1]{{\color{orange} {#1}}} % even even newer
  \newcommand{\newerest}[1]{{\color{blue!10!black!40!green} {#1}}} % you get the idea.
  \newcommand{\ccolor}[1]{{\color{RubineRed}#1}} % Clement
\else
  \newcommand{\new}[1]{{{#1}}}
  \newcommand{\newer}[1]{{{#1}}}
  \newcommand{\newest}[1]{{{#1}}}
  \newcommand{\newerest}[1]{{{#1}}}
  \newcommand{\ccolor}[1]{{#1}}
\fi

\ifnum\withnotes=1
  \newcommand{\cnote}[1]{\par\ccolor{\textbf{C: }\sf #1}} % Clement
  \newcommand{\todonote}[2][]{\todo[size=\scriptsize,color=red!40,#1]{#2}}  
	\newcommand{\questionnote}[2][]{\todo[size=\scriptsize,color=blue!30]{#2}}
	\newcommand{\todonotedone}[2][]{\todo[size=\scriptsize,color=green!40]{$\checkmark$ #2}}
	\newcommand{\todonoteinline}[2][]{\todo[inline,size=\scriptsize,color=orange!40,#1]{#2}}  
  \newcommand{\marginnote}[1]{\todo[color=white,linecolor=black]{{#1}}}
\else
  \newcommand{\cnote}[1]{}
  \newcommand{\todonote}[2][]{\ignore{#2}}
	\newcommand{\questionnote}[2][]{\ignore{#2}}
	\newcommand{\todonotedone}[2][]{\ignore{#2}}
	\newcommand{\todonoteinline}[2][]{\ignore{#2}}
  \newcommand{\marginnote}[1]{\ignore{#1}}
\fi
\newcommand{\ignore}[1]{\leavevmode\unskip} % eat unnecessary spaces before
\newcommand{\cmargin}[1]{\questionnote{\ccolor{#1}}} % Clement

% Shortcuts
\newcommand{\eps}{\ensuremath{\varepsilon}\xspace}
\newcommand{\Algo}{\ensuremath{\mathcal{A}}\xspace} % Algorithm A
\newcommand{\Tester}{\ensuremath{\mathcal{T}}\xspace} % Testing algorithm T
\newcommand{\Learner}{\ensuremath{\mathcal{L}}\xspace} % Learning algorithm L
\newcommand{\property}{\ensuremath{\mathcal{P}}\xspace} % Property P
\newcommand{\class}{\ensuremath{\mathcal{C}}\xspace} % Class C
\newcommand{\eqdef}{\stackrel{\rm def}{=}}
\newcommand{\eqlaw}{\stackrel{\mathcal{L}}{=}}
\newcommand{\accept}{\textsf{ACCEPT}\xspace}
\newcommand{\fail}{\textsf{FAIL}\xspace}
\newcommand{\reject}{\textsf{REJECT}\xspace}
\newcommand{\opt}{{\textsc{opt}}\xspace}
\newcommand{\half}{\frac{1}{2}}
\newcommand{\domain}{\ensuremath{\Omega}\xspace} % Domain of a distribution (default notation)
\newcommand{\distribs}[1]{\Delta\!\left(#1\right)} % Domain of a distribution (default notation)
\newcommand{\yes}{{\sf{}yes}\xspace}
\newcommand{\no}{{\sf{}no}\xspace}
\newcommand{\dyes}{{\cal Y}}
\newcommand{\dno}{{\cal N}}

% Complexity
\newcommand{\littleO}[1]{{o\mleft( #1 \mright)}}
\newcommand{\bigO}[1]{{O\mleft( #1 \mright)}}
\newcommand{\bigOSmall}[1]{{O\big( #1 \big)}}
\newcommand{\bigTheta}[1]{{\Theta\mleft( #1 \mright)}}
\newcommand{\bigOmega}[1]{{\Omega\mleft( #1 \mright)}}
\newcommand{\bigOmegaSmall}[1]{{\Omega\big( #1 \big)}}
\newcommand{\tildeO}[1]{\tilde{O}\mleft( #1 \mright)}
\newcommand{\tildeTheta}[1]{\operatorname{\tilde{\Theta}}\mleft( #1 \mright)}
\newcommand{\tildeOmega}[1]{\operatorname{\tilde{\Omega}}\mleft( #1 \mright)}
\providecommand{\poly}{\operatorname*{poly}}

% Influence
\newcommand{\totinf}[1][f]{{\mathbf{Inf}[#1]}}
\newcommand{\infl}[2][f]{{\mathbf{Inf}_{#1}(#2)}}
\newcommand{\infldeg}[3][f]{{\mathbf{Inf}_{#1}^{#2}(#3)}}

% Sets and indicators
\newcommand{\setOfSuchThat}[2]{ \left\{\; #1 \;\colon\; #2\; \right\} } 			% sets such as "{ elems | condition }"
\newcommand{\indicSet}[1]{\mathds{1}_{#1}}                                              % indicator function
\newcommand{\indic}[1]{\indicSet{\left\{#1\right\}}}                                             % indicator function
\newcommand{\disjunion}{\amalg}%\coprod, \dotcup...

% Distance
\newcommand{\dtv}{\operatorname{d}_{\rm TV}}
\newcommand{\kl}{\operatorname{KL}}
\newcommand{\dhell}{\operatorname{d_{\rm{}H}}}
\newcommand{\hellinger}[2]{{\dhell\mleft({#1, #2}\mright)}}
\newcommand{\kldiv}[2]{{\kl\mleft({#1 \,\|\, #2}\mright)}}
\newcommand{\kolmogorov}[2]{{\operatorname{d_{\rm{}K}}\mleft({#1, #2}\mright)}}
\newcommand{\totalvardistrestr}[3][]{{\dtv^{#1}\mleft({#2, #3}\mright)}}
\newcommand{\totalvardist}[2]{\totalvardistrestr[]{#1}{#2}}
%\newcommand{\chisquarerestr}[3][]{{\operatorname{d}^{#1}_{\chi^2}\mleft({#2 \mid\mid #3}\mright)}}
\newcommand{\chisquare}[2]{{\chi^2\mleft({#1 \mid\mid #2}\mright)}}
\newcommand{\dist}[2]{\operatorname{dist}\mleft({#1, #2}\mright)}

% Restriction (functions, sequences, etc.)
\newcommand\restr[2]{{% we make the whole thing an ordinary symbol
  \left.\kern-\nulldelimiterspace % automatically resize the bar with \right
  #1 % the function
  \vphantom{\big|} % pretend it's a little taller at normal size
  \right|_{#2} % this is the delimiter
  }}

% Probability
\newcommand{\proba}{\Pr}
\newcommand{\probaOf}[1]{\proba\!\left[\, #1\, \right]}
\newcommand{\probaCond}[2]{\proba\!\left[\, #1 \;\middle\vert\; #2\, \right]}
\newcommand{\probaDistrOf}[2]{\proba_{#1}\left[\, #2\, \right]}

% Support of a distribution/function
\newcommand{\supp}[1]{\operatorname{supp}\!\left(#1\right)}

% Expectation & variance
\newcommand{\expect}[1]{\mathbb{E}\!\left[#1\right]}
\newcommand{\expectCond}[2]{\mathbb{E}\!\left[\, #1 \;\middle\vert\; #2\, \right]}
\newcommand{\shortexpect}{\mathbb{E}}
\newcommand{\var}{\operatorname{Var}}

% Distributions
\newcommand{\uniform}{\ensuremath{\mathcal{U}}}
\newcommand{\uniformOn}[1]{\ensuremath{\uniform\!\left( #1 \right) }}
\newcommand{\geom}[1]{\ensuremath{\operatorname{Geom}\!\left( #1 \right)}}
\newcommand{\bernoulli}[1]{\ensuremath{\operatorname{Bern}\!\left( #1 \right)}}
\newcommand{\bern}[2]{\ensuremath{\operatorname{Bern}^{#1}\!\left( #2 \right)}}
\newcommand{\binomial}[2]{\ensuremath{\operatorname{Bin}\!\left( #1, #2 \right)}}
\newcommand{\poisson}[1]{\ensuremath{\operatorname{Poisson}\!\left( #1 \right) }}
\newcommand{\gaussian}[2]{\ensuremath{ \mathcal{N}\!\left(#1,#2\right) }}
\newcommand{\gaussianpdf}[2]{\ensuremath{ g_{#1,#2}}}
\newcommand{\betadistr}[2]{\ensuremath{ \operatorname{Beta}\!\left( #1, #2 \right) }}

% Norms
\newcommand{\norm}[1]{\lVert#1{\rVert}}
\newcommand{\normone}[1]{{\norm{#1}}_1}
\newcommand{\normtwo}[1]{{\norm{#1}}_2}
\newcommand{\norminf}[1]{{\norm{#1}}_\infty}
\newcommand{\abs}[1]{\left\lvert #1 \right\rvert}
\newcommand{\dabs}[1]{\lvert #1 \rvert}
\newcommand{\dotprod}[2]{ \left\langle #1,\xspace #2 \right\rangle } 			% <a,b>
\newcommand{\ip}[2]{\dotprod{#1}{#2}} 			% shortcut

\newcommand{\vect}[1]{\mathbf{#1}} 			% shortcut

% Ceiling and floor
\newcommand{\clg}[1]{\left\lceil #1 \right\rceil}
\newcommand{\flr}[1]{\left\lfloor #1 \right\rfloor}

% Common sets
\newcommand{\R}{\ensuremath{\mathbb{R}}\xspace}
\newcommand{\C}{\ensuremath{\mathbb{C}}\xspace}
\newcommand{\Q}{\ensuremath{\mathbb{Q}}\xspace}
\newcommand{\Z}{\ensuremath{\mathbb{Z}}\xspace}
\newcommand{\N}{\ensuremath{\mathbb{N}}\xspace}
\newcommand{\cont}[1]{\ensuremath{\mathcal{C}^{#1}}}

% Oracles and variants
\newcommand{\ICOND}{{\sf INTCOND}\xspace}
\newcommand{\EVAL}{{\sf EVAL}\xspace}
\newcommand{\CDFEVAL}{{\sf CEVAL}\xspace}
\newcommand{\STAT}{{\sf STAT}\xspace}
\newcommand{\SAMP}{{\sf SAMP}\xspace}
\newcommand{\COND}{{\sf COND}\xspace}
\newcommand{\PCOND}{{\sf PAIRCOND}\xspace}
\newcommand{\ORACLE}{{\sf ORACLE}\xspace}

%% Terminology
\newcommand{\pdfsamp}{dual\xspace}
\newcommand{\cdfsamp}{cumulative dual\xspace}
\newcommand{\Pdfsamp}{\expandafter\capitalisewords\expandafter{\pdfsamp}}
\newcommand{\Cdfsamp}{\expandafter\capitalisewords\expandafter{\cdfsamp}}

% L_p norms
\newcommand{\lp}[1][1]{\ell_{#1}}

% Convolution
\DeclareMathOperator{\convolution}{\ast}

%% Terminology
\newcommand{\D}{\ensuremath{D}}
\newcommand{\distrD}{\ensuremath{\mathcal{D}}}
\newcommand{\birge}[2][\D]{\Phi_{#2}(#1)}
\newcommand{\iid}{i.i.d.\xspace}

% Sign
\DeclareMathOperator{\sign}{sgn}

%% Roman numerals
\makeatletter
\newcommand{\rom}[1]{\romannumeral #1}
\newcommand{\Rom}[1]{\expandafter\@slowromancap\romannumeral #1@}
\newcommand{\century}[2][th]{\Rom{#2}\textsuperscript{#1}}
\makeatother

% Hyperref and \autoref{} -- names
\renewcommand{\sectionautorefname}{Section} % To have "Section 5" instead of "section 5" with \autoref{}
\renewcommand{\chapterautorefname}{Chapter} % To have "Chapter 5" instead of "chapter 5" with \autoref{}
\renewcommand{\subsectionautorefname}{Section} % To have "Section 5" instead of "subsection 5" with \autoref{}
\renewcommand{\subsubsectionautorefname}{Section} % To have "Section 5" instead of "subsubsection 5" with \autoref{}
\def\algorithmautorefname{Algorithm}


%%%%%%%%%%%%%%%%%%%%%%%%%%%%%%%%%%%%%%%%%%%%%%%%%%%%%%%%%%%%%%%%%
% Add author and title info to PDF (and handles multiple authors)
%%%%%%%%%%%%%%%%%%%%%%%%%%%%%%%%%%%%%%%%%%%%%%%%%%%%%%%%%%%%%%%%%
\makeatletter
  \AtBeginDocument{
  \begingroup
  \toks@={}%
  \toksdef\toks@@=2 %
  \toks@@={}%
  \long\def\@ReturnFiFi#1#2\fi\fi{\fi\fi#1}%
  \def\scan@author#1#2 \and#3\@nil{%
  \ifx\\#3\\%
    \ifcase#1 %
      \toks@={#2}%
    \else
      \ifnum#1>1 %
        \toks@=\expandafter{%
          \the\expandafter\toks@\expandafter,\expandafter\space
          \the\toks@@
        }%
      \fi
      \toks@=\expandafter{\the\toks@\space and #2}%
    \fi
    \else
      \ifcase#1 %
        \toks@={#2}%
        \@ReturnFiFi{%
          \scan@author1#3\@nil
        }%
      \else
        \ifnum#1>1 %
          \toks@=\expandafter{%
            \the\expandafter\toks@\expandafter,\expandafter\space
            \the\toks@@
          }%
      \fi
      \toks@@={#2}%
      \@ReturnFiFi{%
        \scan@author2#3\@nil
      }%
    \fi
  \fi
  }%
  \expandafter\expandafter\expandafter\scan@author
  \expandafter\expandafter\expandafter0%
  \expandafter\@author\space\and\@nil
  \edef\x{\endgroup
  \noexpand\hypersetup{pdfauthor={\the\toks@}}%
  }%
  \x
  }
\makeatother

\usepackage{framed}

\title{A ``Twitter thread'' on anticoncentration}
\date{May, 2021}

\usepackage{utopia}
\usepackage{mdframed}
\newmdtheoremenv{quest}{Q}

\renewcommand{\eqdef}{\coloneqq}

\usepackage{soul}

\begin{document}
\begin{flushleft}\sf\footnotesize
\makeatletter
\@date~- \today~(Latest version) \hfill \@title
\makeatother
\end{flushleft}
\vspace{5mm}

\noindent\textbf{tl;dr:} Answers for a \href{https://twitter.com/ccanonne_/status/1395502810804883461}{quiz on anticoncentration held on Twitter}. Overall, the upshot? We cannot say much in most cases; but Paley--Zygmund is a good friend.\bigskip

\hrule
\begin{center}\sc
Questions and Answers
\end{center}
\hrule\medskip

\begin{quest}
Let $X$ be a real-valued random variable with finite expectation $\expect{X}$. What can we say about $\probaOf{X \geq \expect{X}}$?
\end{quest}
\begin{proof}[Answer]
As \newer{55.8\%} of you replied, the answer is a resounding ``not much.'' The probability has to be \hl{strictly positive}, basically by an averaging argument: indeed, suppose that $\probaOf{X \geq \expect{X}}=0$. We can write
$
    \{X \geq \expect{X}\} = \bigcap_{n=1}^\infty \{X > \expect{X} - 1/n\}
$
which implies that, for any fixed $\eps\in(0,1)$, $\probaOf{X > \expect{X} - 1/n} \leq \eps$ for large enough $n$. But then, fixing any $\eps>0$ and any corresponding such large enough $n$, and letting $p \eqdef \probaOf{X > \expect{X} - 1/n} > 0$,
\[
  \expect{X} = \expect{X\indic{X \leq \expect{X} - 1/n}} + \expect{X\indic{X > \expect{X} - 1/n}}
  \leq (\expect{X} - 1/n)\cdot p + \expect{X}\cdot (1-p) <  \expect{X}
\]
contradiction. So $\probaOf{X \geq \expect{X}}>0$.

\noindent However, that's \emph{all} we can say: that probability could be arbitrarily small! Consider, for $n\geq 2$,
\[
  X_n = \begin{cases}
      n & \text{ with probability } \frac{1}{n}\\
      -\frac{n}{n-1} & \text{ with probability } 1-\frac{1}{n}
    \end{cases}
\]
and check that $\expect{X_n}=0$, but $\probaOf{X_n \geq \expect{X_n}} = \frac{1}{n} \xrightarrow[]{n\to\infty} 0$.
\end{proof}

\begin{quest}
Let $X$ be a real-valued r.v. with finite variance $\var[X]$. What can we say about $\probaOf{X \geq \expect{X}}$?
\end{quest}
\begin{proof}[Answer]
This one is really a bummer. It really \emph{feels} like we should be able to say $\probaOf{X \geq \expect{X}} \geq c\cdot \var[X]$, or even $\probaOf{X \geq \expect{X}} \geq c\cdot \sqrt{\var[X]}$ for some absolute constant $c>0$. However, as \newer{47\%} of you answered, we cannot say anything more that for \textbf{Q1}: it's \hl{strictly positive}.

There is a nice counterexample by Iosif Pinellis on \href{https://mathoverflow.net/a/358212/37266}{this MathOverflow answer}, but considering $\probaOf{X > \expect{X}}$; let's modify it a little bit for the case $\probaOf{X \geq \expect{X}}$. First, by replacing $X$ by $X/\sqrt{c\var[X]}$, we can assume the variance is equal to any constant of our choosing, so we'll give something assuming the variance is upper bounded by say $2.1$.\smallskip

\noindent Fix any $n\geq 2$, set $\eps \eqdef \frac{n}{n^2-2}$, and consider $X_n$ given by
\[
  X_n = \begin{cases}
      n & \text{ with probability } \frac{3}{2n^2}\\
      -\eps_n & \text{ with probability } 1-\frac{2}{n^2}\\
      -n & \text{ with probability } \frac{1}{2n^2}
    \end{cases}
\]
If I didn't mess up, we have $\expect{X_n}=0$, $\var[X_n] = 2+ \frac{1}{n^2-2} = 2+o(1)$, but \mbox{$\probaOf{X_n \geq \expect{X_n}} = \frac{3}{2n^2} \xrightarrow[]{n\to\infty} 0$.}
\end{proof}

\begin{quest}
Let $X$ be a real-valued r.v. with finite moments of all orders, and such that $\expect{|X|^k}\leq 1$ for all $k\geq 0$. What can we say about $\probaOf{X \geq \expect{X}}$?
\end{quest}
\begin{proof}[Answer]
Sorry, did I say the \emph{previous} question was a bummer? That one must be the bummest then. I really, really wanted to believe we could say something like $\probaOf{X \geq \expect{X}} \geq c]$ for some absoltue constant $c>0$, but as \newer{34.2\%} of you answered, it's still only as good as \textbf{Q1}: it's \hl{strictly positive}, we cannot say more.

How come? Well, the link to the \href{https://mathoverflow.net/a/358212/37266}{MathOverflow post by Iosif Pinellis} above shows that, given our assumptions (which imply $X \in [-1,1]$ a.s.) we have
\[
    \probaOf{X \geq \expect{X}} \geq \probaOf{X > \expect{X}} \geq \frac{\var[X]}{2}
\]
which, frankly, looked promising (\emph{also, it's a neat proof, check it out!)}. But our assumption is on the raw moments, not the centered ones, so\dots{} $\var[X]$ can still be arbitrarily small (think of $\expect{X^2} \approx \expect{X}^2$, ``when {Jensen} is tight-ish.''). For instance:
fix any $n\geq 1$, and consider $X_n$ given by
\[
  X_n = \begin{cases}
      0 & \text{ with probability } \frac{1}{n}\\
      -1 & \text{ with probability } 1-\frac{1}{n}
    \end{cases}
\]
We have $\expect{X_n}=-(1-\frac{1}{n})$, $\expect{|X_n|^k} \leq 1$ for all $k$, but $\probaOf{X_n \geq \expect{X_n}} = \frac{1}{n} \xrightarrow[]{n\to\infty} 0$.
\end{proof}

\begin{quest}
Let $X$ be a non-negative real-valued r.v. with finite variance. What can we say about $\probaOf{X \geq \frac{1}{2}\expect{X}}$?
\end{quest}
\begin{proof}[Answer]
I have good and bad news. The good news is that there \emph{is} something we can say here. The bad news is that the best option, among those suggested, is still the very disappointing \hl{strictly positive}, as \newer{34.8\%} of you answered.

It cannot be $\probaOf{X \geq \frac{1}{2}\expect{X}}\geq 1/2$, as taking $X$ to be Bernoulli with parameter $p < 1/2$ shows. It cannot be $\probaOf{X \geq \frac{1}{2}\expect{X}}\geq c\cdot \var[X]$ some some absolute constant $c>0$, as the variance could be arbitrarily big, but probabilities tend to be at most one. \emph{(They're stubborn like that.)}

But we \emph{still} can say something! Just something not in the list. Namely, the wonderful--yet--basic--yet--so--useful \emph{Paley--Zygmund inequality}, essentially the single most useful anticoncentration inequality I know, guarantees that for non-negative $X$, letting $\rho(X)\eqdef \frac{\var[X]}{\expect{X}^2}$,
\[
    \probaOf{X > \theta\expect{X}} \geq \frac{(1-\theta)^2}{\rho(X)+(1-\theta)^2}, \quad \theta \in [0,1]
\]
which in our case boilds down to
\[
    \probaOf{X > \frac{1}{2}\expect{X}} \geq \frac{1}{4\rho(X)+1}\,.
\]
Thinking of it differently: ``if the standard deviation and the expectation are comparable, then the random variable cannot be \emph{too} small all the time.''
\end{proof}

\noindent Finally, last question: let's no longer assume $X\geq 0$, and ask for an \emph{anti-Chebyshev}:
\begin{quest}
Let $X$ be a real-valued r.v. with finite variance. What can we say about $\probaOf{|X-\expect{X}| \geq \frac{\sqrt{\var[X]}}{100}}$?
\end{quest}
\begin{proof}[Answer]
  First, recall that Chebyshev's inequality ensures that $\probaOf{|X-\expect{X}| \geq 100\sqrt{\var[X]}} \leq \frac{1}{100^2}$, so we're really asking if some non-trivial converse-type statement holds in general. 
  
  I am so, so sorry. The answer is no, as \newer{40.6\%}. It's \hl{strictly positive}, we cannot say more. One quick and sad way to see it is to consider the non-negative random variable $Y\eqdef (X-\expect{X})^2$, which has $\expect{Y} = \var[X]$ by definition. Then we are asking about 
  \[
    \probaOf{\sqrt{Y} \geq \frac{\sqrt{\expect{Y}}}{100}} = \probaOf{Y \geq \frac{\expect{Y}}{10000}}
  \]
  and then it's clear we cannot say more without extra assumptions on $Y$ (such as an almost sure upper bound, or if we want to use our new friend Paley--Zygmund, some bound on $\var[Y] = \expect{(X-\expect{X}^4}$). For instance, one could take $Y$ to be $0$ with probability $1-1/n$ and $n \expect{Y}$ with probability $1/n$, for arbitrarily large $n$\dots
\end{proof}
\end{document}
